\documentclass[fleqn,reqno,10pt]{article}

%========================================
% Packages
%========================================

\usepackage[nobiblatex]{mypackages}
\usepackage[natbib=true,style=authoryear-comp,backend=biber,doi=false,url=false]{biblatex}
\bibliography{MyRefGlobal}
\usepackage{myenvironments}
\usepackage{mycommands}

%========================================
% Standard Layout
%========================================

% Itemize
\renewcommand{\labelitemi}{\large{$\mathbf{\cdot}$}}    % itemize symbols
\renewcommand{\labelitemii}{\large{$\mathbf{\cdot}$}}
\renewcommand{\labelitemiii}{\large{$\mathbf{\cdot}$}}
\renewcommand{\labelitemiv}{\large{$\mathbf{\cdot}$}}
% Description
\renewcommand{\descriptionlabel}[1]{\hspace\labelsep\textsc{#1}}

% Figure Captions
\usepackage{caption} % use corresponding myfiguresize!
\setlength{\captionmargin}{20pt}
\renewcommand{\captionfont}{\small}
\setlength{\belowcaptionskip}{7pt} % standard is 0pt



\title{Project proposals}
\author{Michael Franke}
\date{last update: \today}

\usepackage[margin = 2.5cm]{geometry}
\newcommand{\scope}[1]{\hfill\textcolor{gray}{[#1]}}

\begin{document}
\maketitle

\section{Alternative-dependent interpretation of probability expressions}

Recent linguistic theories of verbal probability expressions like \emph{probably}, \emph{certainly} or \emph{possibly} have postulated that their meaning is given by a threshold-semantics \citep[e.g.][]{Yalcin2010:Probability-Ope}. For example, \emph{It will probably rain} is true iff the probability of rain according to the speaker is higher than some contextually specified $\theta_{\text{probably}}$. A few studies have looked at the different interpretations of probability expressions \citep[e.g.][]{wind1996}, i.e., trying to determine where exactly a threshold like $\theta_{\text{probably}}$ should be. One case has gained some recent coverage in the online media:
%
\begin{itemize}
\item \url{https://www.vedrashko.com/thisisbig/how-likely-is-probably}
\item \url{https://github.com/zonination/perceptions}
\end{itemize}
%
Building on the latter, it would be interesting to see if/how people's judgements depend on the alternative expressions they encounter in the environment. For example, recent studies from psycholinguistics have shown that which alternative descriptions are present in an experiment can influence the interpretation of certain words \citep[e.g.][]{DegenTanenhaus2012:Processing-Scal,Franke2016:Task-types-link}. This project would experimentally investigate the interpretation of various probability expressions and manipulate the alternative expressions offered to different subjects. This would help to come to understand if the thresholds postulated in the literature are better thought of as pragmatic or semantic.

\scope{experimental, non-technical, course project, BSc thesis} 

\section{Speed-accuracy tradeoffs: diffusion drift or similar models}

There are a number of models which predict choice probalities and reaction times in
2-alternative forced choice tasks, e.g., in perceptual decision making or cardinality
estimation tasks. Examples include the diffusion drift model
\citep[e.g.][]{RatcliffMcKoon2008:The-Diffusion-D} or variants of ACT-R
\citep{AndersonBothell2004:An-Integrated-T}. This project would gather data or use an existing
data set and explore model predictions or compare different models against each other, e.g.,
using existing packages in R (e.g., \url{https://www.biorxiv.org/content/10.1101/570184v1}).

\scope{experimental, modeling, technical, BSc or MSc thesis}

\section{Conduct a mouse-tracking study}

Mouse-tracking is a behavioral method which gives very insightful data concerning a
participants online decision making processes. This project would try to realize a
mouse-tracking study (e.g., for language interpretation  \citep{Jr.Bailey2013:Possibly-all-of,RoettgerStoeber2017:Manual-Response}).

\section{Modeling politeness in language use}

Building on recent models of polite language use (Tessler \& Yoon), this project would try to incorporate the dimensions of politeness outlined in the seminal work by Brown \& Levinson, namely \emph{face threats} and \emph{rate of imposition}, into a Rational Speech Act model of polite language use.

\scope{modeling, linguistics, sociolinguistics, BSc or MSc thesis}

\section{Reference games with topic choice}

Reference games are a widely studied experimental paradigm. In the production part of these experiments, participants are told to select a description of a referent which is highlighted as the topic of conversation in each experimental trial. This project collects data on a related event, namely where the speaker is allowed to freely choose which referent to talk about first, and will then produce a description to communicate to the listener. The hypothesis is that this topic choice is affected by factors such as visual salience, but might also lead to a strategic choice of topic, namely that participants choose expressions for which they have an efficient utterance to communicate it. Finally, it is interesting to compare choices of topics with the results from listener prior elicitation conditions.

\scope{experimental (easy), project or BSc thesis}

\section{Evolution of semantic meaning via pragmatic language use}

Many (agent-based) models of the evolution of semantic meaning do not have a particularly sophisticated vision of how agents use and interpret language when they interact. In response, \citet{BrochhagenFranke2017:Effects-of-tran} have used the Rational Speech Act model to look at simulations addressing the question: which semantic meaning would evolve if agents used language as described by the Rational Speech Act model. In this project, this approach could be applied, for example, to studying the evolution of meaning of gradable adjective, color terms, spatial expressions or a similar interesting domain. This could also use reference games with obects from a multi-dimensional feature space.

\scope{simulation, programming, modeling, language evolution, project, BSc or MSc thesis}

\section{Web-app for (non-)cooperative MasterMind}

We program a web-application that has a little AI and plays a variant of mastermind with our subjects \citep[taking inspiration from][]{VerbruggeMol2008:Learning-to-App}. Subjects see, say, 3 colored objects from a pool of 3 shapes in 3 colors, no repetition possible, but order matters, so we have 504 possible ``world states''. The AI needs to guess the world state. Subjects give feedback on each guess of the AI. Feedback is given by constructing a sentence from a reasonable number of alternative expressions at different positions of the sentence, e.g.:
\begin{itemize}
\item quantifier: {at least one of the, several of the, some of the, all of the, …} (some theoretically interesting ones)
\item object: {squares, blue squares, triangles, green circles, …} (all nine possibilities)
\item predicate: {is/are in the right place, has/have the right color, ...}
\end{itemize}
This can be realized with simple dropdown menus. Feedback must be true (easy and fast to check with a precomputed lookup table). In the cooperative version subjects win if the AI correctly guesses the sequence after, say, 7 rounds. In the uncooperative version subjects win if the AI fails to guess the correct sequence for, say, 10 rounds (the precise numbers need to be chosen so that the game is fun to play, not too easy not too hard, which depends on how smart the AI is). We can measure the informativity of a feedback description quite simply in the program in terms of the number of contextually live possibilities ruled out by the given description at the current state of play. Cooperative language users should tend to choose more informative messages; uncooperative language users should ideally send maximally uninformative, yet true messages. It would be interesting to look at (i) differences in language use between cooperative and uncooperative plays, and possibly (ii) the amount of uncooperativity in choices (measured quantitatively (!) in terms of contextual informativity) as they develop over time.

\scope{programming, web-app, group project or MSc thesis}

\section{Inference of speaker competence}

This project is about how different expressions might lead to different inferences about how knowledgeable the speaker is. For example, hearing Jones say \emph{I ate all of the cake} will not usually have us believe that Jones is in any way uncertain about the amount of cake he ate. This gets more interesting when we come to comparing expressions like \emph{up to 10} to \emph{no more then 10} and \emph{close to 10 but not more} etc. This project would try to collect data using web-based experiments.

\scope{experimental, single project, possibly BSc thesis}

\section{Argumentative language use}

Much linguistic theories on language use assume that the speaker wants to ideally inform the listener, i.e., transfer a maximum of relevant and true information for the sake of the listener's knowledge. However, other forms of discourse exist. An example is \emph{argumentative discourse} where interlocutors hold different opinions about some topic and give arguments in order to persuade one another of their own view \citep[e.g.][]{AnscombreDucrot1983:Largumentation-,MerinInfoRelSocialDecisionmaking1999,GlazerRubinstein2006:A-Game-Theoreti}. This project has two parts which can be independently executed or done in conjunction:
\begin{itemize}
\item In the \textbf{experimental part}, we implement an experiment in which a speaker chooses expressions to influence the hearer of a particular conclusion, e.g., report on the exam results of a class of students \citep[using similar stimuli as][]{Cummins2012:Using-embedded-}.
\item In the \textbf{modeling part}, we extend a model of goal-directed choice of expressions to incorporate a notion of argument strength \citep[e.g., the notion of argumentative relevance of][]{MerinInfoRelSocialDecisionmaking1999}.

\scope{experimental and/or modeling, project BSc and MSc}

\end{itemize}

\section{Bayesian data analysis for a model of hyperbolic language use}

\citet{KaoWu2014:Nonliteral-Unde} suggested an interesting model of how people can interpret non-literal language use. The model is intuitive and backed up by empirical data in the paper. However, the data analysis in the paper leaves some room for exploration. This project would take the data from the original study, and use an implementation of the model to practice a Bayesian data analysis. The goal of this project is to scrutinize the model, i.e., check which aspects of the data the model explains well, and which it might have trouble with.

\scope{Bayesian data analysis, programming, cognitive modeling, individual project or BSc thesis}


\section{Wason selection task and optimal data selection}

The Wason selection task is a classic from the psychology of reasoning. \citet{OaksfordChater1994:A-Rational-Anal} offer a rationalization of participants' choice behavior in terms of \emph{optimal data selection}. This model makes some crucial assumptions which influence the predicted outcome. This project would explore the model by implementing it and changing a number of assumptions. For example, the original model assumes that participants' seek to distinguish between two hypotheses ($p$ and $q$ are probabilistically independent vs.~$p$ makes $q$ likely). There are other possible hypotheses relevant to the interpretation of conditional sentences which arise if we interpret a conditional as an answer to a particular \emph{question under discussion}, such as \emph{What happens if $p$?} vs. \emph{Under which circumstances $q$?}. The project could implement a model and check its predictions with these extended set of participant hypotheses. It could also gather data or use existing data to compare model variants.

\scope{psychology of reasoning, conditionals, linguistics, cognitive modeling, BSc or MSc thesis}

\section{Statistical model comparison for models of vague gradable expressions}

The interpretation of gradable adjectives, like \emph{tall} or \emph{short}, but also of vague quantifiers like \emph{many} and \emph{few}, seems to depend on statistical world knowledge. At least three formal models of how a \emph{prior expectation}, e.g., about the tallness of a person, can lead to choices of adequacy of a description like \emph{John is tall}: a fixed-threshold hypothesis \citep{FernandoKamp1996:Expecting-Many}, an evolutionary account \citep{QingFranke2014:Gradable-Adject}, and a pragmatic reasoning account \citep{LassiterGoodman2015:Adjectival-vagu}. This project would implement a (Bayesian) model comparison of these approaches based on existing experimental data.

\scope{Bayesian data analysis, programming, linguistics, cognitive modeling, project or BSc/MSc thesis}

\section{Preferences for objective  meaning in referential expressions}

Not every word's semantic meaning is equally uncontroversial. What the word \emph{flat} means is less subjective than what \emph{beautiful} means. Speakers of a language should know (at least roughly) which words are more subjective than others. A theory of rational expression choice would predict that speakers would therefore preferably describe a given referent using words with a less subjective meaning (because that minimizes the chance of misunderstanding, when compared to the use of words that have a more subjective meaning; after all, the listener might not agree which of the dogs in front of us is \emph{beautiful}). In this project, we will set up an experiment in which participants choose expressions to refer to objects, such that they choice options are more or less subjective. We will investigate whether the prediction that less subjective expressions are preferred is supported by the data.

\scope{experimental, some modeling, project or BSc thesis (MSc thesis if extended)}

\section{Goodness-of-fit \& model comparison}

There are many measures for a model's goodness-of-fit and several ways of comparing (cognitive)
models. This project would use simulations (based on generating hypothetical empirical data) to
compare different theoretical notions against each other. In particular, this project could
compare correlation scores against likelihood-based criteria.

\scope{statistics, simulation, theory; can be anything: project, thesis etc. depending on scope
and depth}

\section{Optimal stopping (secretary problem)}

The \emph{secretary problem} is a famous instance of a general class of optimal stopping
problems: \url{https://en.wikipedia.org/wiki/Secretary_problem}. While optimal stopping
problems are abstract puzzles which mostly concern theoretical computer science and theoretical
economics, this project would explore optimal stopping from an experimental perspective, a
topic for which substantial and very interesting literature also exists. We would try to cover
novel ground, for instance by having participants learn implicitly the distribution of an
attribute which needs to be maximized. For example, participants are shown 50 boxes with black
and white marbles in sequence. For each box, they can choose it or inspect the next one. They
can never go back but can only choose the current box. The goal is to select the box which
contains with the highest number of marbles from the set of all 50 boxes. (Alternatively, the
goal could be to choose a box with as large a number of marbles as possible.) To play optimally
in this case, participants must screen a number of boxes in order to form beliefs about the
distribution of the number of marbles in the offered boxes. We can model that as rational
belief learning under uncertainty. We will compare such a rational belief learning + rational
choice model to data from an experiment.

\scope{experimental, cognitive modeling, rationality; BSc thesis (ambitious!) or MSc thesis}

\section{Scalar implicatures with probability expressions}

Extend the final model of the second chapter of \url{problang.org} to include also probability
expressions, i.e., yielding complete utterances of the form ``Possibly some of the apples are
red.'' Additionally, experimental data could be collected to test the model's predictions.

\scope{pragmatics, webppl, programming, experimental; project, possibly BSc thesis}

\section{Imprecision (semantic slack) in RSA}

Semantic slack is the phenomenon that we often interpret expressions which have a precise
meaning as true even if they are, strictly speaking, false
\citep[e.g.][]{Bach1994:Semantic-Slack:,Lasersohn1999:Pragmatic-Halos,Krifka2002:Be-Brief-and-Va,Krifka2007:Approximate-Int}.
This project would try to incorporate this phenomenon into a probabilistic pragmatics model. A
first step into this direction has been made by \citet{KaoWu2014:Nonliteral-Unde} where a model
is presented that integrates the possibility of interpreting a number words as imprecisely
used. This project could build on this model (which is also presented in Chapter III or
\url{problang.org}) but include reasoning about multiple levels of precision. 

\scope{pragmatics, webppl, programming, linguistics, philosophy of language; BSc/MSc thesis}

\section{Efficient learning from good teachers}

\citet{CsibraGergely2011:Natural-Pedagog} argue that human children are born with a unique
skill set which evolved to make them particularly good learners. Part of this package of skill
is the ability to infer information about the intentions of a teacher who produces good examples
\citep{ShaftoGoodman2014:A-rational-acco}. This project could look at rational
(probabilistically) learner models, either in iterated learning models
\citet{KirbyGriffith2014:Iterated-Learni}, or as models of word learning
\citep{FrankGoodman2014:Inferring-word-}. 

\scope{developmemntal psychology, learning, concepts, language acquisition, probabilistic
  modeling; BSc/MSc thesis}

\section{Free-choice inferences in probabilistic pragmatics}

Free-choice inferences are inferences associated, for example, with sentences like in
(\ref{bsp:FC_trigger}), which naturally obtain a reading like in (\ref{bsp:FC_inference}), even
though this does not follow from their standard logical semantics
\cite{KampFreeChoice1973,KampFreeChoice1978,ZimmermannFreeChoiceDisjunction2000,Fox2007:Free-Choice-and}. 

\begin{exe}
  \ex
  \begin{xlist}
    \ex \label{bsp:FC_trigger} You may have cake or ice-cream.
    \ex \label{bsp:FC_inference} $\leadsto$ You may have a cake and you may have
    ice-cream (but possibly not both at the same time).
  \end{xlist}
\end{exe}

While a lot of work has been devoted to explaining free-choice inference in formal pragmatics,
a convincing account in the tradition of Rational Speech Act models
\cite{FrankGoodman2012:Predicting-Prag} is missing. The project would explore different
possibilities (e.g., in terms of reasoning about lexical uncertainty
\cite{BergenLevy2014:Pragmatic-Reaso}) for explaining free-choice inferences in RSA.

\scope{linguistics, pragmatics, probabilistic modeling; BSc/MSc thesis}


\section{Biases in Probabilistic Reasoning}

There are several well-known systematic ``mistakes'' humans are prone to make in probabilistic
reasoning. This project could look at a selection of these mistakes and summarize the
literature. For example, the project could look at two seemingly opposing phenomena, namely the
``gambler's fallacy'' and ignorance of what is called ``regression to the mean'', and search
the literature for explanations or models of human reasoning that explain why humans are
susceptible to these mistakes. 

\scope{human reasoning, cognitive psychology, rationality, probability; BSc thesis}


\section{Frequency of Implicature Answers in a Truth-Value Judgement Task}

A scalar implicature is an inference from a sentence like ``John ate some of the cookies'' to
the conclusion that ``John did not eat all of the cookies''. This inference is thought to come
about by pragmatic reasoning about what the speaker could have said but did not
\citep[e.g.][]{Geurts2010:Quantity-Implic}. There is an ongoing controversy about how these
implicatures arise during language processing. One interesting finding is that the degree /
frequency to which scalar implicatures arises depends on how much cognitive resources
participants have. For example, in a dual-task paradigm, where participants also had to solve
another task, the rate of scalar implicature endorsement was lower the more additional
resources the secondary task depleted \citep{NeysDe-NeysSchaeken2007:When-People-Are}. Using a
truth-value judgement task borrowed from \citet{Noveck2001:When-Children-a},
\citet{FrankeDulcinati2020:Strategies-of-d} observed in small pilot studies that just adding a
short pause before participants can answer whether a sentence like ``Some elephants are
mammals'' is true influences the rate at which participants give pragmatic answers (here:
saying ``false'' because all elephants are mammals). This project would systematically
investigate this dependency of scalar implicature endorsement on response-enabling latencies in
an online or laboratory experiment. 

\scope{linguistics, pragmatics, psycholinguistics, experiments; project or BSc thesis}

\section{Explanatory considerations in updating}

\citet{DouvenSchupbach2015:The-role-of-exp} report on three experiments purported to show how a
notion of ``explanatoriness'' of a hypothesis is useful for explaining participants judgements
of posterior probability on top of Bayesian posterior updating. The experiments are potentially
flawed because the two critical measures are always recorded one after another. The possibility
that given one judgement first might influence the other is real. This project could probe into
this issue, replicate one of the three experiments (ideally Experiment 3) and then run a
follow-up in which participants are not subsequently asked for both ``explanatoriness'' and
``probability'' judgements. The experiment(s) can easily be realized as online experiments.

\scope{philosophy, probabilistic reasoning, experiments; project or BSc thesis}

\section{Probabilistic pragmatic model for modified numerals}

Modified numerals are expressions such as \emph{at most three}, \emph{less than five},
\emph{about twenty} or \emph{between six and eight}. They are used to communicate quantities
with more or less certainty about the true value. There is a substantial literature on modified
numerals from the point of view of formal semantics
\citep[e.g.][]{GeurtsNouwen2007:At-least-et-al:,Nouwen2010:Two-Kinds-of-Mo,CoppockBrochhagen2013:Raising-and-Res,Kennedy2015:A-de-Fregean-se}.
In this project, you would extend a Rational Speech Act model
\citep{FrankGoodman2012:Predicting-Prag,FrankeJager2015:Probabilistic-p,ScontrasTessler2018:Probabilistic-L}
of a probabilistic pragmatic speaker (who is either certain of the true state or not) and who
chooses among many different modified and unmodified numeral expressions (possibly also
non-numeric quantifiers such as \emph{some}, \emph{most}, \emph{few}) to communicate their
beliefs about that quantity to a listener.

\scope{pragmatics, probabilistic modeling, BSc or MSc thesis}

\printbibliography[heading=bibintoc]

\end{document}
